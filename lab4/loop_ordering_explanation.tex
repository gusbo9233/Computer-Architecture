\documentclass{article}
\usepackage{listings}
\usepackage{xcolor}
\usepackage{geometry}

\geometry{margin=1in}

\lstset{
    basicstyle=\ttfamily,
    keywordstyle=\color{blue},
    commentstyle=\color{green!60!black},
    numbers=left,
    numberstyle=\tiny,
    numbersep=5pt,
    frame=single,
    breaklines=true
}

\title{Improving Loop Ordering in MIPS Assembly}
\author{Loop Optimization Analysis}
\date{\today}

\begin{document}

\maketitle

\section{Overview}
This document explains the modifications made to improve the loop ordering in a MIPS assembly program that sums array elements. The original program used an inefficient memory access pattern, while the modified version implements row-major ordering for better performance.

\section{Computer Architecture Background}

\subsection{Memory Hierarchy}
Modern computer systems implement a memory hierarchy to bridge the speed gap between fast processors and relatively slow main memory:
\begin{itemize}
    \item CPU registers (fastest, smallest)
    \item L1 cache (very fast, small)
    \item L2 cache (fast, medium)
    \item Main memory (slow, large)
\end{itemize}

\subsection{Cache Organization}
\begin{itemize}
    \item Data is transferred between memory levels in fixed-size blocks called cache lines
    \item Typical cache line sizes range from 32 to 128 bytes
    \item When a byte is accessed, the entire cache line containing it is brought into cache
\end{itemize}

\subsection{Locality Principles}
Two key principles govern efficient memory access:
\begin{itemize}
    \item Temporal Locality: Recently accessed items are likely to be accessed again soon
    \item Spatial Locality: Items physically stored close to each other tend to be accessed close together in time
\end{itemize}

\subsection{Memory Access Patterns}
\begin{itemize}
    \item Sequential access (stride-1): Best pattern, maximizes spatial locality
    \item Strided access: Less efficient, may cause more cache misses
    \item Random access: Worst pattern, minimal locality
\end{itemize}

\subsection{Row-Major vs Column-Major Order}
In multi-dimensional arrays:
\begin{itemize}
    \item Row-major order (used in C/C++): Elements in a row are stored contiguously
    \item Column-major order (used in Fortran): Elements in a column are stored contiguously
    \item Accessing arrays in the same order as they are stored in memory maximizes cache efficiency
\end{itemize}

\subsection{Impact on Performance}
Poor memory access patterns can lead to:
\begin{itemize}
    \item Cache misses: CPU must wait for data to be fetched from slower memory
    \item Cache pollution: Useful data being evicted by less frequently used data
    \item Memory bandwidth wastage: Loading cache lines that are only partially used
    \item TLB (Translation Lookaside Buffer) misses: Additional overhead in virtual memory translation
\end{itemize}

\section{Key Changes}

\subsection{Loop Structure}
The original program had a complex three-level nested loop structure:
\begin{itemize}
    \item An outer loop that counted to 10 iterations
    \item A middle loop that handled index offsets in increments of 4
    \item An inner loop that accessed memory in strides of 16 bytes
\end{itemize}

The new version simplifies this to two loops:
\begin{itemize}
    \item An outer loop that counts 10 iterations
    \item An inner loop that sequentially accesses every element
\end{itemize}

\subsection{Memory Access Pattern}
\subsubsection{Original Version}
The original code:
\begin{itemize}
    \item Used a stride of 16 bytes (addi \$t0, \$t0, 16)
    \item Required additional index calculations (\$t6)
    \item Accessed memory in a non-sequential pattern
\end{itemize}

\subsubsection{Improved Version}
The new code:
\begin{itemize}
    \item Uses sequential access with 4-byte increments (addi \$t0, \$t0, 4)
    \item Eliminates complex index calculations
    \item Follows natural memory layout (row-major order)
\end{itemize}

\section{Register Usage Optimization}
\begin{itemize}
    \item Removed \$t6 and \$t7 registers that were used for complex indexing
    \item Simplified register usage to:
    \begin{itemize}
        \item \$t0: Current array address
        \item \$t1: End address
        \item \$t2: Running sum
        \item \$t3: Current array element
        \item \$t4: Iteration counter
        \item \$t5: Maximum iterations
    \end{itemize}
\end{itemize}

\section{Performance Benefits}

\subsection{Cache Efficiency}
The new version provides better cache utilization because:
\begin{itemize}
    \item Sequential memory access patterns are more cache-friendly
    \item Reduces cache misses by following spatial locality principles
    \item Better prefetching opportunities for the hardware
\end{itemize}

\subsection{Computational Efficiency}
The improved version:
\begin{itemize}
    \item Reduces the number of instructions executed
    \item Eliminates unnecessary address calculations
    \item Simplifies control flow logic
\end{itemize}

\section{Conclusion}
The modifications result in a more efficient program that:
\begin{itemize}
    \item Is easier to understand and maintain
    \item Makes better use of the memory hierarchy
    \item Reduces computational overhead
    \item Follows best practices for array traversal
\end{itemize}

The improved version maintains the same functionality while providing better performance characteristics through proper row-major ordering and simplified control flow.

\section{Cache Performance Analysis}

\subsection{Cache Configuration Analysis}
Let's analyze how different cache configurations affect the hit rate. For all calculations:
\begin{itemize}
    \item Total array size = 1024 bytes
    \item Each element = 4 bytes
    \item Program reads each element 10 times
    \item Total memory accesses = (1024/4) × 10 = 2560 accesses
\end{itemize}

\subsection{Original Configuration (512B cache, 4B blocks)}
\begin{itemize}
    \item Number of blocks = 512/4 = 128 blocks
    \item First access to each element is a miss
    \item Subsequent 9 accesses could be hits if the block remains in cache
    \item Total misses = 256 (one miss per element)
    \item Hit rate = (2560 - 256)/2560 = 0.9 or 90\%
\end{itemize}

\subsection{512B cache, 8B blocks}
\begin{itemize}
    \item Number of blocks = 512/8 = 64 blocks
    \item Each block contains 2 elements
    \item First access to each block is a miss
    \item Total misses = 128 (one miss per block)
    \item Hit rate = (2560 - 128)/2560 = 0.95 or 95\%
    \item Improvement due to spatial locality: when one element is loaded, its neighbor is also loaded
\end{itemize}

\subsection{512B cache, 16B blocks}
\begin{itemize}
    \item Number of blocks = 512/16 = 32 blocks
    \item Each block contains 4 elements
    \item First access to each block is a miss
    \item Total misses = 64 (one miss per block)
    \item Hit rate = (2560 - 64)/2560 = 0.975 or 97.5\%
    \item Further improvement due to better spatial locality exploitation
\end{itemize}

\subsection{1024B cache, 4B blocks}
\begin{itemize}
    \item Number of blocks = 1024/4 = 256 blocks
    \item Can hold entire array in cache
    \item Misses only on first access to each element
    \item Total misses = 256 (one miss per element)
    \item Hit rate = (2560 - 256)/2560 = 0.9 or 90\%
    \item Same hit rate as 512B cache because block size is the limiting factor
\end{itemize}

\subsection{1024B cache, 8B blocks}
\begin{itemize}
    \item Number of blocks = 1024/8 = 128 blocks
    \item Each block contains 2 elements
    \item Total misses = 128 (one miss per block)
    \item Hit rate = (2560 - 128)/2560 = 0.95 or 95\%
    \item Improvement due to spatial locality
\end{itemize}

\subsection{1024B cache, 16B blocks}
\begin{itemize}
    \item Number of blocks = 1024/16 = 64 blocks
    \item Each block contains 4 elements
    \item Total misses = 64 (one miss per block)
    \item Hit rate = (2560 - 64)/2560 = 0.975 or 97.5\%
    \item Best performance due to largest block size
\end{itemize}

\subsection{Perfect Hit Rate Analysis}
A perfect hit rate of 1.0 cannot be achieved without modifying the program because:
\begin{itemize}
    \item First access to any memory location will always be a miss
    \item Cold misses (compulsory misses) are unavoidable
    \item Even with a cache large enough to hold the entire array:
    \begin{itemize}
        \item First iteration will have misses
        \item Subsequent iterations will be hits
        \item Best possible hit rate = (Total accesses - First access misses)/Total accesses
    \end{itemize}
    \item To achieve a perfect hit rate, the program would need to:
    \begin{itemize}
        \item Prefetch all data before it's needed
        \item Or implement a warm-up phase before the actual computation
    \end{itemize}
\end{itemize}

\subsection{Key Observations}
\begin{itemize}
    \item Larger block sizes improve hit rate due to spatial locality
    \item Cache size matters less than block size for this access pattern
    \item Sequential access pattern helps maximize benefit of larger blocks
    \item Cold misses prevent achieving perfect hit rate without program modification
\end{itemize}

\section{Cache Performance Comparison}

\subsection{Part 1: Original Program Analysis (Bad Loop Ordering)}
With the original program using non-sequential access:

\subsubsection{512B Cache, 4B Blocks, Direct Mapping}
\begin{itemize}
    \item Hit rate = 0\%
    \item Total misses = 2560 (all accesses miss)
    \item Reason: Strided access pattern (16-byte stride) causes conflict misses
    \item Each element maps to the same cache location due to direct mapping
    \item Even though cache can hold 128 blocks, stride causes continuous eviction
\end{itemize}

\subsubsection{512B Cache, 8B Blocks, Direct Mapping}
\begin{itemize}
    \item Hit rate remains 0\%
    \item Compulsory misses reduced to 320 (2560/8)
    \item Still suffers from conflict misses due to strided access
    \item Larger block size doesn't help because access pattern skips blocks
\end{itemize}

\subsubsection{512B Cache, 4B Blocks, 2-Way Set Associative}
\begin{itemize}
    \item Hit rate remains 0\%
    \item Even with set associativity, stride causes conflicts
    \item Access pattern defeats both spatial locality and set associativity benefits
\end{itemize}

\subsection{Part 2: Optimized Program Analysis (Row-Major Order)}

\subsubsection{512B Cache, 4B Blocks, Direct Mapping}
\begin{itemize}
    \item Hit rate = 90\%
    \item Total accesses = 2560 (256 elements × 10 iterations)
    \item Compulsory misses = 256 (one per unique element)
    \item Computation: (2560 - 256)/2560 = 0.9
    \item Significant improvement due to sequential access pattern
\end{itemize}

\subsubsection{512B Cache, 8B Blocks, Direct Mapping}
\begin{itemize}
    \item Hit rate = 95\%
    \item Compulsory misses = 128 (one per 8B block)
    \item Computation: (2560 - 128)/2560 = 0.95
    \item Improved due to better spatial locality exploitation
\end{itemize}

\subsubsection{512B Cache, 16B Blocks, Direct Mapping}
\begin{itemize}
    \item Hit rate = 97.5\%
    \item Compulsory misses = 64 (one per 16B block)
    \item Computation: (2560 - 64)/2560 = 0.975
    \item Further improvement from larger blocks
\end{itemize}

\subsubsection{1024B Cache Analysis}
For 4B blocks:
\begin{equation*}
    \text{Hit Rate} = \frac{\text{Total Accesses} - \text{Compulsory Misses}}{\text{Total Accesses}} = \frac{2560 - 256}{2560} = 0.9
\end{equation*}

For 8B blocks:
\begin{equation*}
    \text{Hit Rate} = \frac{2560 - 128}{2560} = 0.95
\end{equation*}

For 16B blocks:
\begin{equation*}
    \text{Hit Rate} = \frac{2560 - 64}{2560} = 0.975
\end{equation*}

\subsection{Perfect Hit Rate Analysis}
A perfect hit rate (1.0) cannot be achieved because:
\begin{itemize}
    \item Compulsory misses are unavoidable for first access
    \item Even with optimal:
    \begin{itemize}
        \item Sequential access pattern
        \item Large enough cache (1024B)
        \item Larger block sizes (16B)
    \end{itemize}
    \item Best achievable rate is 97.5\% with current parameters
    \item Would need prefetching or warm-up phase to eliminate compulsory misses
\end{itemize}

\subsection{Key Findings}
\begin{itemize}
    \item Row-major ordering dramatically improves hit rate from 0\% to 90-97.5\%
    \item Larger block sizes benefit sequential access pattern
    \item Cache size less important than access pattern and block size
    \item Compulsory misses set upper limit on achievable hit rate
\end{itemize}

\end{document} 