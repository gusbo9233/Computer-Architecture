\documentclass{article}\\
\usepackage{amsmath}\\
\usepackage{geometry}\\

\geometry{a4paper, margin=1in}\\

\begin{document}\\

\section*{Part 1}\\

Configure the Cache Size to 512 bytes, the Block Size to 4 bytes, the Mapping to ``direct mapping'' and run the program. Look at the statistics and answer the following questions:\\

\begin{enumerate}\\
    \item What is the hit rate? Why do we see this value?\\
    
    Hit rate = 0\%\\
    
    The program jumps 16 bytes each time.\\
    When we try to reuse data, it's already been replaced\\
    because the same locations keep getting used.\\

    \item How many misses do we have in total? How many of them are compulsory misses? Explain your answer.\\
    
    Total misses = 2560\\
    Compulsory misses = 256\\
    Conflict misses = 2304\\
    
    Our array has 256 numbers in it.\\
    We access each number 10 times, so 2560 total accesses.\\
    First time we read each number, it must miss (256 misses).\\
    Every other time misses too because data keeps getting replaced.\\

    \item Change the block size from 4 bytes to 8 bytes but keep the cache size to 512 bytes (reduce the number of blocks accordingly). How are the compulsory misses affected when Block Size is changed from 4 bytes to 8 bytes? Is there any change, why/why not?
    
    Hit rate improves to 50\%\\
    
    Now each block can hold two numbers instead of one.\\
    This means we need fewer blocks to load all the data.\\

    \item Change the block size back to 4 again, and change the mapping from ``direct mapping'' to ``2-ways set associative''? Cache size should be 512 bytes again, change the number of blocks accordingly. Is there any change, why/why not?
    
    Hit rate = 0\%\\
    
    Even with two choices for where to put data,\\
    we still keep replacing the numbers we need\\
    because we jump by 16 bytes each time.\\

    \item With the block size of 4, experiment with different cache sizes by changing the number of blocks. Try with both ``direct mapping'' and ``2-ways set associative''. How the hit rate is improved? At which point changing the cache size does not affect the hit rate?
    
    Cache size = 512 bytes (128 blocks): 0\% hit rate \\
    Cache size = 1024 bytes (256 blocks): 90\% hit rate \\
    Cache size = 2048 bytes (512 blocks): 90\% hit rate \\
    
    For 2-way set associative:\\
    Cache size = 512 bytes (128 blocks): 0\% hit rate \\
    Cache size = 1024 bytes (256 blocks): 90\% hit rate \\
    Cache size = 2048 bytes (512 blocks): 90\% hit rate \\

    I did not see the hit rate in the simulator when increasing to 256 blocks but it should only be initial misses \\
    so I assume it is 90\% hit rate. \\ 
    Hit rate gets better at 1024 bytes because the 256 blocks are enough to hold all our data.\\
    Making it bigger doesn't help anymore.\\

    \item At this ``optimal'' cache size (question 5) with the ``direct mapping'' option, what hit rates do you obtain by changing the block size? Try \& report the hit rates with the sizes 4 bytes, 8 bytes and 16 bytes (you should keep cache size the same).\\
    
    4-byte blocks: 90\% hit rate\\
    8-byte blocks: 95\% hit rate\\
    16-byte blocks: 97.5\% hit rate\\

    \item Show how the achieved hit rate (as shown by simulator) can be computed (by hand) for different block sizes for the ``optimal'' cache size you found in question 5. Hint: what is the size of a single element in the array?\\
    
    Given:\\
    Cache size = 1024 bytes (optimal from question 5)\\
    Array size = 256 words (1024 bytes)\\
    Element size = 1 word (4 bytes)\\
    Number of elements = 256\\
    Program jumps: 16 bytes each time (addi \$t0, \$t0, 16)\\
    Each element accessed 10 times\\
    Total accesses = 256 × 10 = 2560\\
    
    Direct Mapping:\\
    
    For 1-word blocks:\\
    Cache size = 1024 bytes\\
    Each block can hold 1 number\\
    We have 256 blocks\\
    First time loading needs 256 misses\\
    Because we jump 16 bytes, we keep missing\\
    All 2560 accesses miss\\
    Hit rate = (2560 - 2560)/2560 = 0\%\\
    
    For 2-word blocks:\\
    Cache size = 1024 bytes\\
    Each block can hold 2 numbers\\
    We have 128 blocks\\
    First time loading needs 128 misses\\
    We jump by 16 bytes (every 2nd block)\\
    After first load, numbers stay in cache\\
    Total misses = 128\\
    Hit rate = (2560 - 128)/2560 = 95\%\\
    
    For 4-word blocks:\\
    Cache size = 1024 bytes with 64 blocks\\
    Each block can hold 4 numbers\\
    We have 64 blocks\\
    First time loading needs 64 misses\\
    Our 16-byte jumps match block size perfectly\\
    Numbers stay in cache after first load\\
    Total misses = 64 (only compulsory misses remain)\\
    Hit rate = (2560 - 64)/2560 = 0.975 = 97.5%

\end{enumerate}

\section*{Part 2}

Rewrite the program in such a way that all the elements in the array are accessed sequentially i.e. in a good loop ordering or row major ordering. Configure the Cache Size to 512 bytes, the Block Size to 4 bytes, the Mapping to ``direct mapping'' and run the program. Look at the statistics and answer the following questions.

\begin{enumerate}
    \item Does the hit rate improve? Why? (show computation)
    
    Computation:\\
    Cache size = 512 bytes\\
    Block size = 8 bytes (2 words)\\
    Number of blocks = 512/8 = 64 blocks\\
    Each block can hold 2 numbers in a row\\
    We need 128 blocks for all 256 numbers\\
    Total accesses = 256 × 10 = 2560\\
    Our cache can only hold half the blocks we need:\\
    First access to all blocks: 128 misses\\
    Need to reload blocks each iteration: 128 misses\\
    Total misses = 128 + (9 × 128) = 1280\\
    Hit rate = (2560 - 1280)/2560 = 50\%

    \item Change the Block Size to 16 bytes (keep the cache size the same, 512 bytes). Does the hit rate improve? Why? (show computation)
    
    Computation:\\
    Cache size = 512 bytes\\
    Block size = 16 bytes (4 words)\\
    Number of blocks = 512/16 = 32 blocks\\
    Each block can hold 4 numbers in a row\\
    We need 64 blocks for all 256 numbers\\
    Total accesses = 2560\\
    First time loading needs 64 misses\\
    Need to reload everything each time: 64 misses\\
    Total misses = 64 + (9 × 64) = 640\\
    Hit rate = (2560 - 640)/2560 = 75\%

    \item Change the Cache Size to 1024 bytes and Reset the Block Size to 8 bytes. What is the hit rate? Compute this hit rate by hand.
    
    Computation:\\
    Cache size = 1024 bytes\\
    Block size = 8 bytes (2 words)\\
    Number of blocks = 1024/8 = 128 blocks\\
    Each block can hold 2 numbers in a row\\
    We need 128 blocks for all numbers\\
    Total accesses = 2560\\
    Cache is big enough for all blocks\\
    Only miss first time loading: 128 misses\\
    Total misses = 128\\
    Hit rate = (2560 - 128)/2560 = 95\%

    \item Change Block Size to 16 bytes (cache size is 1024 bytes now). What is the hit rate? Compute this hit rate by hand.
    
    Computation:\\
    Cache size = 1024 bytes\\
    Block size = 16 bytes (4 words)\\
    Number of blocks = 1024/16 = 64 blocks\\
    Each block can hold 4 numbers in a row\\
    We need 64 blocks for all numbers\\
    Total accesses = 2560\\
    Cache is big enough for all blocks\\
    Only miss first time loading: 64 misses\\
    Total misses = 64\\
    Hit rate = (2560 - 64)/2560 = 97.5\%

    \item Can a perfect hit rate of 1.0 be achieved without changing the program? Why?
    
    Perfect hit rate cannot be achieved because we always \\
    miss the first access to each block. Larger block sizes\\
    improve hit rate by reducing number of compulsory misses. This is assuming \\
    that the cache is empty when the program starts.

\end{enumerate}

\end{document} 